% !TeX spellcheck = de_DE
\documentclass[a4paper,11pt]{ctexart}

\usepackage{setspace}
\usepackage{titlesec}
\usepackage{titletoc}
\usepackage[shortlabels]{enumitem}
\usepackage[hmargin=1.25in,vmargin=1in]{geometry}
\usepackage{amsmath}
\usepackage{amssymb}
\usepackage{indentfirst}
\usepackage[colorlinks,linkcolor=blue,anchorcolor=blue,citecolor=green]{hyperref}
\usepackage{tikz}
\usetikzlibrary{arrows.meta}
\usetikzlibrary{bending}
\usetikzlibrary{math}
\usetikzlibrary{shapes.symbols}
\usetikzlibrary{shapes.geometric}
\usetikzlibrary{shapes.arrows}
\usepackage[american inductors]{circuitikz}
\usepackage{booktabs}
\usepackage{array}
\usepackage{multirow}
\usepackage{upgreek}
\usepackage{gensymb}
\usepackage{xfrac}
\usepackage{subfig}
\usepackage{float}
\usepackage{placeins}
\usepackage{url}
\usepackage{lscape}
\usepackage{rotating}
\usepackage{graphicx}
\graphicspath{{graph/}}
\usepackage{pythonhighlight}

\renewcommand\theequation{\arabic{equation}}
\renewcommand{\contentsname}{}
\renewcommand\arraystretch{1.5}

\newcommand{\kV}{\,\text{kV}}
\newcommand{\V}{\,\text{V}}
\newcommand{\mV}{\,\text{mV}}
\newcommand{\uV}{\,\text{\mu V}}
\newcommand{\MOhm}{\,\text{M}\Omega}
\newcommand{\kOhm}{\,\text{k}\Omega}
\newcommand{\Ohm}{\,\Omega}
\newcommand{\mOhm}{\,\text{m}\Omega}
\renewcommand{\H}{\,\text{H}}
\newcommand{\uH}{\,\text{\mu H}}
\newcommand{\nH}{\,\text{nH}}
\newcommand{\s}{\,\text{s}}
\newcommand{\ms}{\,\text{ms}}
\newcommand{\us}{\,\text{\mu ms}}
\newcommand{\ns}{\,\text{ns}}
\newcommand{\kA}{\,\text{kA}}
\newcommand{\A}{\,\text{A}}
\newcommand{\mA}{\,\text{mA}}
\newcommand{\uA}{\,\text{\mu A}}
\newcommand{\MW}{\,\text{MW}}
\newcommand{\kW}{\,\text{kW}}
\newcommand{\W}{\,\text{W}}
\newcommand{\mW}{\,\text{mW}}
\newcommand{\MVar}{\,\text{MVar}}
\newcommand{\MVA}{\,\text{MVA}}
\newcommand{\VA}{\,\text{VA}}
\renewcommand{\S}{\,\text{S}}
\newcommand{\km}{\,\text{km}}
\newcommand{\m}{\,\text{m}}
\newcommand{\cm}{\,\text{cm}}
\newcommand{\mm}{\,\text{mm}}
\newcommand{\diff}{\text{d}}
\newcommand{\Hz}{\,\text{Hz}}
\newcommand{\kHz}{\,\text{kHz}}
\newcommand{\MHz}{\,\text{MHz}}
\newcommand{\rpm}{\text{rpm}}
\newcommand{\T}{\text{T}}
\newcommand{\Wb}{\text{Wb}}
\newcommand{\kg}{\text{kg}}
\renewcommand{\j}{\text{j}}
\newcommand{\ssum}{\scriptscriptstyle \sum}


%\renewcommand{\omega}{\upomega}

\newcommand{\du}[1]
{
	#1^{\circ}
}
\newcommand{\ang}[1]
{
	\angle#1^{\circ}
}

\newcommand{\bfem}[1]
{
	\em\bfseries#1\normalfont
}

\newcommand{\subpar}
{
	\par
	\hangafter = 0
	\setlength{\hangindent}{1em}
}

\newcommand{\subsubpar}
{
	\par
	\hangafter = 0
	\setlength{\hangindent}{2em}
}
\newcommand{\subsubsubpar}
{
	\par
	\hangafter = 0
	\setlength{\hangindent}{3em}
}

\newenvironment{shrinkeq}[2]
{
	\bgroup
	\addtolength\abovedisplayshortskip{#1}
	\addtolength\abovedisplayskip{#1}
	\addtolength\belowdisplayshortskip{#2}
	\addtolength\belowdisplayskip{#2}
}
{
	\egroup
	\ignorespacesafterend
}

\setcounter{secnumdepth}{4}
\newcounter{designitem}
\setcounter{designitem}{0}
\newcommand{\entry}
{
	\vspace{0.5em}
	\par
	\stepcounter{designitem}
	\thedesignitem.
}
\title
{
	\linespread{1.5} \zihao{4}
	高压数字测量系统课程设计 \\ 
	\zihao{2}
	开题报告
}
\author
{
	谢弘洋
}
\date{}

\def\Z2{28}

\begin{document}
	\pagestyle{plain}
	
\begin{figure}[t]
	\setlength{\abovecaptionskip}{-10mm}
	\setlength{\belowcaptionskip}{-60mm}
	\centering
	\includegraphics[scale=0.4]{page1.png}
\end{figure}

\begin{center}
	\zihao{-3}
	电\,气\,工\,程\,及\,其\,自\,动\,化\,-\,电\,机\,设\,计 \\
	\vspace{0.7em}
	
	\zihao{1}
	课\hspace{0.5em} 程\hspace{0.5em} 作\hspace{0.5em} 业\\
	\vspace{3em}
	\zihao{-3}
	\renewcommand\arraystretch{1.2}
	
	\begin{tabular}{>{\centering\arraybackslash}p{5em}p{1em}>{\centering\arraybackslash}p{7em}}
		\begin{tabular}{>{\centering\arraybackslash}p{5em}}
			姓\hspace{1em}名\\
			\midrule
			熊\hspace{0.25em}承\hspace{0.25em}清\\
			周\hspace{0.25em}修\hspace{0.25em}宁\\
			谢\hspace{0.25em}弘\hspace{0.25em}洋
		\end{tabular}&
		\begin{tabular}{p{1em}}
			\\
			\\
			\\

		\end{tabular}&
		\begin{tabular}{>{\centering\arraybackslash}p{7em}}
			学\hspace{1em}号\\
			\midrule
			5130xxxxxxxx\\
			5130xxxxxxxx\\
			515021910641
	\end{tabular}
		
		
	\end{tabular}
	
	
	\vspace{6em}
	\today
\end{center}

\newpage
\begin{spacing}{1.5}
	\tableofcontents
\end{spacing}

\titleformat{\section}{\Large\bfseries\raggedright}{(\chinese{section})}{5pt}{}
\titlespacing{\section}{0pt}{10pt}{5pt}
\titleformat{\subsection}{\bfseries\zihao{-4}}{\arabic{subsection}.}{5pt}{}
\titlespacing{\subsection}{1em}{2pt}{3pt}
\titleformat{\subsubsection}{\bfseries\normalsize}{\arabic{subsection}.\arabic{subsubsection}}{5pt}{}
\titlespacing{\subsubsection}{2em}{1pt}{0pt}
\titleformat{\paragraph}{\bfseries\normalsize}{\arabic{subsection}.\arabic{subsubsection}.\arabic{paragraph}}{5pt}{}
\titlespacing{\paragraph}{3em}{1pt}{0pt}

\newpage
\begin{spacing}{1.5}
	
	\zihao{-4}
	
\section*{设计题目}
\addcontentsline{toc}{section}{设计题目}
\par
已知数据:输出功率$P_N = 0.75\kW$,电压$U_N = 380\V$(星型接法),定子电流$I_1 = 1.81\A$,机械转速$n = 2830\rpm$,相数$m_1 = 3$,频率$f = 50\Hz$,极对数$p = 2$,效率$\eta = 75\%$,功率因数$\cos\varphi^{'} = 0.84$,堵转转矩与额定转矩之比$\sfrac{T_{st}}{T_{N}} = 2.2$,B级绝缘,连续运行,封闭型自扇冷式,主要性能指标按技术条件JB3074-82的规定。

\section{额定数据和主要尺寸}
\entry
额定功率$P_{N} = 0.75\kW$
\entry
额定电压$U_{N} = 380\V,\quad U_{N\phi} = 220\V$(星型接法)
\entry
功电流$I_{KW} = \frac{P_N}{m_1U_{N\phi}} = \frac{0.75\times10^3}{3\times220}\A=1.136\A $
\entry
效率$\eta^{'}$
\par
按照设计要求取$\eta^{'} = 75\%$
\entry
功率因数$\cos\varphi^{'}$
\par
按照设计要求取$\cos\varphi^{'} = 0.84$
\entry
极对数$p = 1$
\entry
定转子槽数
\par
每极每相槽数为整数。参考类似规格电机$q_1 = 3$,则$Z_1 = 2m_1pq_1 = 2\times3\times1\times2\times3 = 18$。再按表10-8选$Z_2 = 16$,并采用转子斜槽。
\entry
定转子每极槽数
\begin{shrinkeq}{-1ex}{-1ex}
	\begin{align}
	Z_{p1} &= \frac{Z_1}{2p} = \frac{18}{2} = 9\\
	Z_{p2} &= \frac{Z_2}{2p} = \frac{16}{2} = 8
	\end{align}
\end{shrinkeq}
\entry
确定电机主要尺寸
\par
一般可参考类似电机的主要尺寸来确定$D_{i1}$和$l_{ef}$。现按\textsection10-2中的分析来确定。
\par
由经验公式可得满载电势标幺值
\begin{shrinkeq}{-1ex}{-1ex}
	\begin{align}
	K_{B}^{'} &= 0.0108\ln P_N - 0.013p +0.931 \\
	&= 0.0108\ln 0.75 - 0.013\times 1 +0.931 = 0.915 \notag
	\end{align}
\end{shrinkeq}
\par
由式(10-7)可求出计算功率
\begin{shrinkeq}{-1ex}{-1ex}
	\begin{align}
	P^{'} &= K_{B}^{'}\frac{P_N}{\eta^{'}\cos\varphi^{'}} \\
	&= 0.915\times\frac{0.75\times 10^3}{0.75\times 0.84} = 1.089\times 10^3 \VA\notag
	\end{align}
\end{shrinkeq}
\par
初选$\alpha_{p}^{'} = 0.68$,$K_{Nm}^{'} = 1.11$,$K_{dp1}^{'} = 0.92$,由图10-2取$A^{'} =18000\sfrac{\A}{\m}$,由表10-1取$B_{\delta}^{'} = 0.6\T$,根据设计要求$n^{'}=2830\rpm$,于是由式(10-4)得
\begin{shrinkeq}{-1ex}{-1ex}
	\begin{align}
	V &= \frac{6.1}{\alpha_{p}^{'}K_{Nm}^{'}K_{dp1}^{'}}\cdot\frac{1}{A^{'}B_{\delta}^{'}}\cdot\frac{P^{'}}{n^{'}} \\
	&=\frac{6.1}{0.68\times 1.11\time 0.92}\times\frac{1}{18000\times 0.6}\times\frac{1.089\times 10^3}{2830}\m^3 \notag\\
	&=0.000313 \notag
	\end{align}
\end{shrinkeq}
\par
由表10-2取$\lambda = 0.75$,代入式(10-9),得
\begin{shrinkeq}{-1ex}{-1ex}
	\begin{align}
	D_{i1}^{'} &= \sqrt[3]{\frac{2p}{\lambda\pi}V} \\
	&=\sqrt[3]{\frac{2}{0.75\pi}\times 0.000313}\m = 0.0643\m \notag
	\end{align}
\end{shrinkeq}
再由表10-3,按定子内外径比求出定子冲片外径
\begin{shrinkeq}{-1ex}{-1ex}
	\begin{align}
	D_{1}^{'} = \frac{D_{i1}^{'}}{\sfrac{D_{i1}}{D}} = \frac{0.0643}{0.56}  = 0.115\m
	\end{align}
\end{shrinkeq}
\par
根据标准直径最后确定$D_1 = 0.12\m$。于是
\begin{shrinkeq}{-1ex}{-1ex}
	\begin{align}
	D_{i1} = D_1\times\frac{D_{i1}}{D_1} = 0.12\times 0.56\m  \approx 0.0672\m
	\end{align}
\end{shrinkeq}
\par
铁心的有效长度
\begin{shrinkeq}{-1ex}{-1ex}
	\begin{align}
	l_{ef} = \frac{V}{D_{i1}^{2}} = \frac{0.000313}{0.0672^2} = 0.0693\m
	\end{align}
\end{shrinkeq}
取铁心长度$l_{i}=0.065\m$。(按照生产要求,铁心长度通常采用$5\mm$进位)。
\entry
气隙的确定
\par
参考类似产品或由经验公式(10-10a),得
\begin{shrinkeq}{-1ex}{-1ex}
	\begin{align}
	\delta &= 0.3\left(0.4+7\sqrt{D_{i1}l_i}\right)\times 10^{-3} \\
	&=0.3\times\left(0.4+7\sqrt{0.0.0672\times 0.065}\right)\times 10^{-3}\m = 0.259\times 10^{-3} \notag
	\end{align}
\end{shrinkeq}
\par
于是铁心有效长度$l_{ef}=l_i+2\delta = \left(0.065+2\times 0.259\times 10^{-3}\right) = 0.0655\m$
\par
转子外径$D_2 = D_{i1}-2\delta = \left(0.0672-2\times 0.259\times 10^{-3}\right) = 0.0667\m$
\par
转子内径先按转轴直径决定(以后再校验转子轭部磁密):$D_{i2} = 0.026\m$

\entry
极距
\begin{shrinkeq}{-1ex}{-1ex}
	\begin{align}
	\uptau = \frac{\pi D_{i1}}{2p} = \frac{\pi\times 0.0672}{2\times 1}\m = 0.0106\m
	\end{align}
\end{shrinkeq}
\entry
定子齿距
\begin{shrinkeq}{-1ex}{-1ex}
	\begin{align}
	t_1 = \frac{\pi D_{i1}}{Z_1} = \frac{\pi\times 0.0672}{18}\m = 0.01173\m
	\end{align}
\end{shrinkeq}
\par
\hspace{1.75em}转子齿距
\begin{shrinkeq}{-1ex}{-1ex}
	\begin{align}
	t_2 = \frac{\pi D_2}{Z_2} = \frac{\pi\times 0.0667}{16}\m = 0.0131\m
	\end{align}
\end{shrinkeq}
\entry
定子绕组采用双层绕组,叠绕组。
\entry
为了削弱齿谐波磁场的影响,转子采用斜槽,一般斜一个定子齿距$t_1$,于是转子斜槽宽$b_{sk} = 0.0115\m$。
\entry
设计定子绕组
\par
按照式(10-14),每相串联导体数
\begin{shrinkeq}{-1ex}{-1ex}
	\begin{align}
	N_{\phi 1}^{'} &= \frac{\eta^{'}\cos\varphi^{'}\pi D_{i1}A^{'}}{m_1I_{KW}}\\
	&=\frac{0.75\times 0.84\times\pi\times 0.0672\times 18000}{3\times 1.136} = 702 \notag
	\end{align}
\end{shrinkeq}
取并联支路数$a_1 = 1$,由式(10-15),可得每槽导体数
\begin{shrinkeq}{-1ex}{-1ex}
	\begin{align}
	N_{s1}^{'} &= \frac{m_1a_1N_{\phi 1}^{'}}{Z_1}\\
	&=\frac{3\times 1 \times 702}{18} = 117\notag
	\end{align}
\end{shrinkeq}
取$N_{s1} = 118$,于是每线圈匝数为59。
\entry
每相串联导体数
\begin{shrinkeq}{-1ex}{-1ex}
	\begin{align}
	N_{\phi 1} = \frac{N_{s1}Z_1}{m_1a_1} = \frac{118 \times 18}{3\times 1} = 708 \notag
	\end{align}
\end{shrinkeq}
\par
\hspace{1.75em}每相串联匝数
\begin{shrinkeq}{-1ex}{-1ex}
	\begin{align}
	N_{1} &= \frac{N_{\phi 1}}{2} = \frac{708}{2} =  354\notag
	\end{align}
\end{shrinkeq}
\entry
绕组线规设计
\par
初选定子电密$J_{1}^{'} = 6\sfrac{\A}{\mm^2}$,由式(10-16),计算导线绕组并绕根数和每根导线截面积的乘积。
\begin{shrinkeq}{-1ex}{-1ex}
	\begin{align}
	N_{i1}A_{c1}^{'} = \frac{I_{1}^{'}}{a_1J_{1}^{'}} = \frac{1.803}{1\times 6}\mm^2 = 0.3005\mm^2
	\end{align}
\end{shrinkeq}
其中定子电流初步估计值
\begin{shrinkeq}{-1ex}{-1ex}
	\begin{align}
	I_{1}^{'} = \frac{I_{KW}}{\eta^{'}\cos\varphi^{'}} = \frac{1.136}{0.75\times 0.84}\A = 1.803\A
	\end{align}
\end{shrinkeq}
在附录二中选用截面积相近的铜线:高强度漆包线,并绕根数$N_{i1} =1 $,线径$d_1 = 0.63\mm$,绝缘后直径$d = 0.68\mm$,截面积$A_{c1}^{'} = 0.3117\mm^2$,$N_{i1}A_{c1}^{'} = 0.3117\mm^2$。
\entry
设计定子槽形
\par
因定子绕组为圆导线散嵌,故采用梨形槽,齿部平行。初步取$B_{i1}^{'} = 1.6\T$,按式(10-18),估计定子齿宽
\begin{shrinkeq}{-1ex}{-1ex}
	\begin{align}
	b_{i1} = \frac{t_1B_{\delta}^{'}}{K_{Fe}B_{i1}^{'}} = \frac{0.01173\times 0.6}{0.95 \times 1.4}\m = 0.00463\m
	\end{align}
\end{shrinkeq}
初步取$B_{j1}^{'} = 1.5\T$,按式(10-19),估计定子轭部计算高度
\begin{shrinkeq}{-1ex}{-1ex}
	\begin{align}
	h_{j1}^{'} = \frac{\uptau\alpha_{p}^{'}B_{\delta}^{'}}{2K_{Fe}B_{j1}^{'}} = \frac{0.106\times 0.6\times 0.68}{2\times 0.95\times 1.5}\m = 0.0152\m
	\end{align}
\end{shrinkeq}
按齿宽和定子轭部计算高度的估算值做出定子槽形如图 ,槽口尺寸参考类似产品决定,取$b_{01} =2.5 \mm$,$h_{01} = 0.5\mm$。齿宽计算如下:
\begin{shrinkeq}{-1ex}{-1ex}
	\begin{align}
	b_{i1} &= \frac{\pi\left(D_{i1}+2h_{01}+2h_{11}+2h_{21}\right)}{Z_1}-2r_{21}\\
	&=\left[\frac{\pi\left(0.0672+2\times 0.0005+2\times 0.0015+2\times 0.0059\right)}{18}-2\times 0.0049\right]\m \notag\\
	&= 0.00469\m\notag
	\end{align}
\end{shrinkeq}
\begin{shrinkeq}{-1ex}{-1ex}
	\begin{align}
	b_{i1} &= \frac{\pi\left(D_{i1}+2h_{01}+2h_{11}\right)}{Z_1}-b_{11}\\
	&= \left[\frac{\pi\left(0.0672+2\times 0.0005+2\times 0.0015\right)}{36} -0.0078 \right]\m \notag\\
	&= 0.004627\m \notag
	\end{align}
\end{shrinkeq}
齿部基本平行,齿宽$b_{i1} = 0.00463\m$(平均值)。
\entry
槽满率
\par
槽面积
\begin{shrinkeq}{-1ex}{-1ex}
	\begin{align}
	A_{s} &= \frac{2r_{21}+b_{11}}{2}\left(h_{s}^{'}-h\right)+\frac{\pi r_{21}^{2}}{2}\\
	&=\left[\frac{2\times 0.0049+0.0078 }{2}\times\left( 0.0079-0.002 \right)+\frac{\pi\times 0.0049^2}{2}\right]\m^2\notag\\
	&= 8.606\times 10^{-5}\m^2\notag
	\end{align}
\end{shrinkeq}
\par
按附录三,槽绝缘采用DMDM复合绝缘,$\Delta_{i} = 0.25\mm$,槽楔为$h = 2\mm$层压板,槽绝缘占面积$A_{i} = \Delta_{i}\left(2h_{s}^{'}+\pi r_{21}+2r_{21}+b_{11}\right) = 0.00025\times\left(2\times 0.0079+\pi\times 0.0049+2\times 0.0049+ 0.0078\right)\m^2 = 1.199\times 10^{-5}\m^2$
\par
槽有效面积
\begin{shrinkeq}{-1ex}{-1ex}
	\begin{align}
	A_{ef} = A_{s}-A{i} = \left(8.606\times 10^{-5} -1.199times 10^{-5} \right)\m^2 =7.406\times 10^{-5} \m^2
	\end{align}
\end{shrinkeq}
\par
槽满率
\begin{shrinkeq}{-1ex}{-1ex}
	\begin{align}
	s_{f} &= \frac{N_{i1}N_{s1}d^{2}}{A_{ef}}\\
	& = \frac{118\times\left(0.68\times 10^{-3}\right)^2}{7.406\times 10^{-5} } = 73.67\% \qquad\text{符合要求}\notag
	\end{align}
\end{shrinkeq}
\entry
绕组系数
\begin{shrinkeq}{-1ex}{-1ex}
	\begin{align}
	\beta = \frac{y}{Z_{p1}}=\frac{y}{m_1q_1} = \frac{5}{6}\\
	K_{p1} = \sin\frac{\beta\pi}{2} = \sin\frac{5\pi}{12} = 0.966
	\end{align}
\end{shrinkeq}
\begin{shrinkeq}{-1ex}{-1ex}
	\begin{align}
	K_{d1} = \frac{\sin\frac{q\alpha}{2}}{q\sin\frac{\alpha}{2}} = \frac{\sin\frac{3\times\du{20}}{2}}{3\sin\frac{\du{20}}{2}} = 0.960
	\end{align}
\end{shrinkeq}
其中
\begin{shrinkeq}{-1ex}{-1ex}
	\begin{align}
	\alpha = p\frac{2\pi}{Z_1}=\frac{1\times\du{360}}{18} = \du{20}
	\end{align}
\end{shrinkeq}
\begin{shrinkeq}{-1ex}{-1ex}
	\begin{align}
	K_{dp1} = K_{d1}K_{p1} = 0.960\times 0.966 = 0.927
	\end{align}
\end{shrinkeq}
\par
每相有效串联导体数
\begin{shrinkeq}{-1ex}{-1ex}
	\begin{align}
	N_{\phi 1}K_{dp1} = 708\times 0.927 = 656
	\end{align}
\end{shrinkeq}

\entry
设计转子槽形和转子绕组
\par
按式(10-39),预计转子导条电流:
\begin{shrinkeq}{-1ex}{-1ex}
	\begin{align}
	I_{2}^{'} = K_{I}I_{1}^{'}\frac{3N_{\phi 1}K_{dp1}}{Z_2} =0.89\times 1.803\times \frac{3\times 708\times 0.927}{16}\A = 197\A
	\end{align}
\end{shrinkeq}
其中,$K_{I}$由表10-10查出。
\par
初步取转子导条电密$J_{B}^{'} = 3.5\sfrac{\A}{\mm^2}$,于是导条截面积
\begin{shrinkeq}{-1ex}{-1ex}
	\begin{align}
	A_{B}^{'}=\frac{I_{2}^{'}}{J_{B}^{'}} = \frac{197}{3.5}\mm^2 = 56\mm^2
	\end{align}
\end{shrinkeq}
\par
初步取$B_{i2}^{'} = 1.3\T$,估算转子齿宽
\begin{shrinkeq}{-1ex}{-1ex}
	\begin{align}
	b_{i2}^{'} = \frac{t_2B_{\delta}^{'}}{K_{Fe}B_{i2}^{'}}=\frac{0.0131\times 0.6}{0.95\times 1.3}\m = 0.00636\m
	\end{align}
\end{shrinkeq}
初步取$B_{j2}^{'} = 1.25\T$,估算转子轭部计算高度
\begin{shrinkeq}{-1ex}{-1ex}
	\begin{align}
	b_{j2}^{'} = \frac{\uptau \alpha_{p}^{'}B_{\delta}^{'}}{2K_{Fe}B_{j2}^{'}}=\frac{0.106\times 0.68\times 0.6}{2\times 0.95\times 1.25}\m = 0.0182\m
	\end{align}
\end{shrinkeq}
为获得较好的起动性能,采用平行槽,作槽形图如图 所示,取槽口尺寸$b_{02} = 0.6\mm$,$h_{02} = 0.3\mm$。

\par
齿壁不平行的槽形的齿宽(按\textsection 3-3)计算如下:
\begin{shrinkeq}{-1ex}{-1ex}
	\begin{align}
	b_{i2_{\frac{1}{8}}} &= \frac{\pi\left[D_2-2\times\frac{2}{3}\left(h_{02}+h_{12}+h_{22}\right)\right]}{Z_2}-b_{12} \\
	&=\left[\frac{\pi\times\left(0.0667 - 2\times\frac{2}{3}\left(0.0003+0.0016+0.0047\right)  \right)}{16}-0.0061 \right]\m\notag\\ 
	&= 0.00527\m \notag
	\end{align}
\end{shrinkeq}
导条截面积(转子槽面积)
\begin{shrinkeq}{-1ex}{-1ex}
	\begin{align}
	A_{B} &= \frac{b_{02}+b_{12}}{2}h_{12}+b_{12}h_{22}+\frac{\pi r_{22}^{2}}{2} \\
	&=\left[\frac{0.0006+0.0061}{2}\times 0.0016 +0.0061 \times 0.0047 +
	\frac{\pi\times 0.0031^2}{2}\right]\m^2 \notag\\
	&= 4.91\times 10^{-5}\m^2\notag
	\end{align}
\end{shrinkeq}
\par
按式(11-41)估计端环电流
\begin{shrinkeq}{-1ex}{-1ex}
	\begin{align}
	I_{R}^{'}=I_{2}^{'}\frac{Z_2}{2\pi p} = 197\times \frac{16}{2\pi\times 1}\A = 501\A
	\end{align}
\end{shrinkeq}
\par
端环所需面积
\begin{shrinkeq}{-1ex}{-1ex}
	\begin{align}
	A_{R}^{'} = \frac{I_{R}^{'}}{J_{R}^{'}} = \frac{501}{0.6\times 3.5}\mm^2 = 238\mm^2
	\end{align}
\end{shrinkeq}
其中,端环电密$J_{R}^{'} = 0.6 J_{B}^{'} =2.1 \sfrac{\A}{\mm^2}$。

\section{磁路计算}
\entry
计算满载电势
\par
初设$K_{B}^{'}=1-\epsilon_{L}^{'} = 0.927$,由式(3-51),得
\begin{shrinkeq}{-1ex}{-1ex}
	\begin{align}
	E_1 = \left(1-\epsilon_{L}^{'}\right)U_{N\phi} = 0.927\times 220\V = 203.94\V
	\end{align}
\end{shrinkeq}
\entry
计算每极磁通
\par
初设$K_{s}^{'} = 1.15$,由图3-5查得$K_{Nm} = 1.0975$,由式(3-9),得
\begin{shrinkeq}{-1ex}{-1ex}
	\begin{align}
	\Phi = \frac{E_1}{4K_{Nm}K_{dp1}fN_1} = \frac{203.94}{4\times 1.0975\times 0.927\times 50\times 354}\Wb = 0.00283\Wb
	\end{align}
\end{shrinkeq}
\par
为计算磁路各部分磁密,需要先计算磁路中各部分得导磁截面:
\entry
每极下齿部截面积
\begin{shrinkeq}{-1ex}{-1ex}
	\begin{align}
	A_{i1} &= K_{Fe}l_{i}b_{i1}Z_{p1} = 0.95\times 0.065\times 0.00463\times 9\m^2 = 2573\times 10^{-6}\m^2\\
	A_{i1} &= K_{Fe}l_{i}b_{i2}Z_{p2} = 0.95\times 0.065\times 0.00517\times 8\m^2 = 2554\times 10^{-6}\m^2
	\end{align}
\end{shrinkeq}
\entry
定子轭部计算高度由式(3-37)
\begin{shrinkeq}{-1ex}{-1ex}
	\begin{align}
	h_{j1}^{'}&=\frac{D_1-D_{i1}}{2}-h_{s1}+\frac{r_{21}}{3}\\
	&=\left[\frac{0.12 - 0.0672}{2}-\left( 0.0005+ 0.0015+ 0.0059+0.0049\right)\times 10^{-3}+\frac{ 0.0049\times 10^{-3}}{3}\right]\m \notag\\
	&= 0.0152\m \notag
	\end{align}
\end{shrinkeq}
\par
转子轭部计算高度由式(3-38)
\begin{shrinkeq}{-1ex}{-1ex}
	\begin{align}
	h_{j2}^{'}&=\frac{D_2-D_{i2}}{2}-h_{s2}+\frac{r_{22}}{3}\\
	&=\left[\frac{0.0667 - 0.019}{2}-\left( 0.001613+ 0.0047+ 0.0003+0.00309\right)\times 10^{-3}+\frac{ 0.00309\times 10^{-3}}{3}\right]\m \notag\\
	&= 0.01517\m \notag
	\end{align}
\end{shrinkeq}
轭部导磁截面积
\begin{shrinkeq}{-1ex}{-1ex}
	\begin{align}
	A_{j1} &= K_{Fe}l_{i}h_{j1}^{'} = 0.95\times 0.065\times 0.0152 = 938.6\times 10^{-6}\m^2 \\
	A_{j2} &= K_{Fe}l_{i}h_{j2}^{'} = 0.95\times 0.065\times 0.01517 = 936.7\times 10^{-6}\m^2 
	\end{align}
\end{shrinkeq}
\entry
一极下空气隙截面积
\begin{shrinkeq}{-1ex}{-1ex}
	\begin{align}
	A_{\delta} = \uptau l_{ef} = 0.106\times 0.0655\m^2 = 6943\times 10^{-6}\m^2
	\end{align}
\end{shrinkeq}
\entry
磁路计算所选的是通过磁极中心线的闭合回路(见\textsection3-1),该回路上的气隙磁密是最大值$B_{\delta}$(见\textsection3-2)。为此,由图3-5,先找出计算极弧系数$\alpha_{p}^{'}= $,由此求得波幅系数
\begin{shrinkeq}{-1ex}{-1ex}
	\begin{align}
	F_{s} = \frac{B_{\delta}}{B_{\delta\text{av}}} = \frac{1}{\alpha_{p}^{'}} =1.4925 
	\end{align}
\end{shrinkeq}
按照“程序”,$F_s$可由图按附1-1按初设的$K_{s}^{'}$查取,但须注意,为了实用上的方便,“程序”中$F_{s}$的定义与这里的略有不同,并且计算每极磁通$\Phi$的公式也做了相应的改变,详见[3-3]。
\entry
气隙磁密由式(3-7)计算
\begin{shrinkeq}{-1ex}{-1ex}
	\begin{align}
	B_{\delta} = \frac{F_{s}\Phi}{A_{\delta}} = \frac{ 1.4925\times 0.00283}{ 6943\times 10^{-6}}\T = 0.6083\T
	\end{align}
\end{shrinkeq}
\entry
由式(3-26),对应与气隙磁密最大hi处的定子齿部磁密
\begin{shrinkeq}{-1ex}{-1ex}
	\begin{align}
	B_{i1} &= \frac{B_{\delta}l_{ef}t_1}{K_{Fe}l_{i}^{i}b_{i1}}\cdot\frac{Z_{p1}}{Z_{p1}} = \frac{F_{s}\Phi}{A_{i1}} \\
	&=\frac{1.4925\times 0.00283}{ 2573\times 10^{-6}}\T = 1.642\T
	\end{align}
\end{shrinkeq}
\entry
转子齿部磁密
\begin{shrinkeq}{-1ex}{-1ex}
	\begin{align}
	B_{i2} = \frac{F_{s}\Phi}{A_{i2}} = \frac{1.4925 \times 0.00283}{ 2554\times 10^{-6}}\T = 1.654\T
	\end{align}
\end{shrinkeq}
\entry
从附录五的D23磁化曲线找出对应上述磁密的磁场强度
\begin{shrinkeq}{-1ex}{-1ex}
	\begin{align}
	H_{i1} = 50\sfrac{\A}{\cm};\quad H_{i2} = 53.4\sfrac{\A}{\cm}
	\end{align}
\end{shrinkeq}
\entry
有效气隙长度
\begin{shrinkeq}{-1ex}{-1ex}
	\begin{align}
	\delta_{ef} = K_{\delta}\delta = 1.032\times 0.2588\times 10^{-3}\m = 0.2671\times 10^{-3}\m
	\end{align}
\end{shrinkeq}
\par
其中气隙系数按式(3-20)计算
\begin{shrinkeq}{-1ex}{-1ex}
	\begin{align}
	K_{\delta 1} &= \frac{t_1\left(4.4\delta+0.75 b_{01}\right)}{t_1\left(4.4\delta +0.75 b_{01}\right)-b_{01}^{2}} \\
	& = \frac{0.01173\times\left(4.4\times 0.2588+0.75\times 0.5\right)}{0.01173\times\left(4.4\times 0.2588+0.75\times 0.5\right) - 0.5^2\times 10^{-3}} = 1.0143\notag\\
	K_{\delta 2} &= \frac{t_2\left(4.4\delta+0.75 b_{02}\right)}{t_2\left(4.4\delta +0.75 b_{02}\right)-b_{02}^{2}} \\
	& = \frac{0.0131\times\left(4.4\times 0.2588+0.75\times 0.6\right)}{0.0131\times\left(4.4\times 0.2588+0.75\times 0.6\right) - 0.6^2\times 10^{-3}} = 1.0176\notag\\
	K_{\delta} &= K_{\delta 1}K_{\delta 2} = 1.0143\times 1.0176 =1.032 
	\end{align}
\end{shrinkeq}
\entry
齿部磁路计算长度按式(3-28)计算
\begin{shrinkeq}{-1ex}{-1ex}
	\begin{align}
	L_{i1} &= \left(h_{11}+h_{21}\right)+\frac{1}{3}r_{21} \\
	&= \left(0.0015+0.059 +\frac{0.0049}{3}\right) = 9.03\times 10^{-3}\m \notag\\
	L_{i2} &= \left(h_{12}+h_{22}\right)+\frac{1}{3}r_{22} \\
	&= \left(0.001613 +0.0047 +\frac{0.00309}{3}\right) = 7.343\times 10^{-3}\m \notag
	\end{align}
\end{shrinkeq}
\entry
按式(3-40)计算轭部磁路计算长度
\begin{shrinkeq}{-1ex}{-1ex}
	\begin{align}
	L_{j1}^{'} &=\frac{\pi\left(D_1-h_{j1}^{'}\right)}{2p}\times\frac{1}{2} = \frac{\pi\left(0.12 - 0.0152\right)}{2\times 1\times 2} = 0.0823\m\\
	L_{j2}^{'} &=\frac{\pi\left(D_{i2}+h_{j2}^{'}\right)}{2p}\times\frac{1}{2} = \frac{\pi\left(0.019 + 0.01517\right)}{2\times 1\times 2} = 0.0268\m
	\end{align}
\end{shrinkeq}
\entry
按式(3-6)计算气隙磁压降
\begin{shrinkeq}{-1ex}{-1ex}
	\begin{align}
	F_{\delta} = \frac{K_{\delta}B_{\delta}\delta}{\mu_0} = \frac{1.032 \times 0.6083\times 0.2588}{1.25\times 10^{-6}} = 129.973\A
	\end{align}
\end{shrinkeq}
\entry
齿部磁压降
\begin{shrinkeq}{-1ex}{-1ex}
	\begin{align}
	F_{i1} &= H_{i1}L_{i1} = 50\times 10^{2}\times 9.03\times 10^{-3} \A = 45.15\A \\
	F_{i2} &= H_{i2}L_{i2} = 53.4\times 10^{2}\times 7.343\times 10^{-3} \A = 39.21\A 
	\end{align}
\end{shrinkeq}
\entry
饱和系数按(3-15)计算
\begin{shrinkeq}{-1ex}{-1ex}
	\begin{align}
	K_{s} = \frac{F_{\delta}+F_{i1}+F_{i2}}{F_{\delta}} = \frac{129.973 +45.15 + 39.21}{129.973} = 1.65
	\end{align}
\end{shrinkeq}
与初设值$K_{s}^{'} = 1.15$相比较,误差$\sfrac{\left( 1.65- 1.15\right)}{1.65} = 30.3\%$太大。
\par
计算出的$K_{s} = 1.65$,比原假设值$K_{s}^{'}$大,说明原假设的$K_{s}^{s}$偏低,在此基础上计算出的气隙磁密最大值$B_{}\delta$和齿部磁密$B_{i1}$、$B_{i2}$都偏大,致使计算出的$K_{s}$高于实际值。再次假设时取$K_{s}^{'}<K_{s}^{'\!'}<K_{s}$。为使计算结果能够较快收敛,常按经验公式取$K_{s}^{'\!'} = K_{s}-\sfrac{\left(K_{s}-K_{s}^{'}\right)}{3}$。
\par
假设$K_{s}^{'\!'} =1.4 $,重新计算第23$\sim$37项中有关各式:
\setcounter{designitem}{22}
\entry
$K_{s}^{'\!'} = 1.4$,由图3-5查得$K_{Nm} = 1.085$;$\Phi = 0.00286$
\setcounter{designitem}{26}
\entry
$\alpha_{p}^{'} = 0.704$,$F_{s} = 1.42$
\entry
$B_{\delta} = 0.5850\T$
\entry
$B_{i1} = 1.5783\T$
\entry
$B_{i2} = 1.5901\T$
\entry
$H_{i1} = 35.1\sfrac{\A}{\cm}$;$H_{i2} = 34\sfrac{\A}{\cm}$
\setcounter{designitem}{34}
\entry
$F_{\delta} = 125.6\A$
\entry
$F_{i1} = 29.08\A$;$F_{i2} = 25.6\A$
\entry
$K_{s} = 1.435$。误差$\sfrac{\left( 1.435- 1.4\right)}{1.435} = 2.43\%\qquad$仍大。

\par
再次假设$K_{s}^{'\!'\!'} =1.42 $,重新计算第23$\sim$37项中有关各式:
\setcounter{designitem}{22}
\entry
$K_{s}^{'\!'\!'} = 1.42$,由图3-5查得$K_{Nm} = 1.086$;$\Phi = 0.00286$
\setcounter{designitem}{26}
\entry
$\alpha_{p}^{'} = 0.706$,$F_{s} = 1.417$
\entry
$B_{\delta} = 0.5837\T$
\entry
$B_{i1} = 1.575\T$
\entry
$B_{i2} = 1.587\T$
\entry
$H_{i1} = 32.2\sfrac{\A}{\cm}$;$H_{i2} = 31.5\sfrac{\A}{\cm}$
\setcounter{designitem}{34}
\entry
$F_{\delta} = 125.3\A$
\entry
$F_{i1} = 28.44\A$;$F_{i2} = 24.82\A$
\entry
$K_{s} = 1.425$。误差$\sfrac{\left( 1.425- 1.42\right)}{1.425} = 0.35\%\qquad$合格。

\entry
定子轭部磁密按式(3-36)计算
\begin{shrinkeq}{-1ex}{-1ex}
	\begin{align}
	B_{j1} = \frac{1}{2}\cdot\frac{\upphi}{A_{j1}} = \frac{1}{2}\times\frac{0.00286 }{939\times 10^{-6}} = 1.523\T
	\end{align}
\end{shrinkeq}
\entry
转子轭部磁密
\begin{shrinkeq}{-1ex}{-1ex}
	\begin{align}
	B_{j2} = \frac{1}{2}\cdot\frac{\upphi}{A_{j2}} = \frac{1}{2}\times\frac{0.00286}{937\times 10^{-6} } = 1.526\T
	\end{align}
\end{shrinkeq}
\entry
从附录五的D23磁化曲线找出对应上述磁密的磁场强度:$H_{j1} = 22.8\sfrac{\A}{\cm}$;$H_{j2} = 23.1\sfrac{\A}{\cm}$
\entry
按式(3-42)计算轭部磁压降,其中轭部磁压降校正系数见图附1-3b。
\begin{shrinkeq}{-1ex}{-1ex}
	\begin{align}
	\frac{h_{j1}^{'}}{\uptau} = \frac{0.0152}{0.106} = 0.1434,\qquad B_{j1} = 1.523 \T\qquad\text{于是}C_{j1} = 0.42
	\end{align}
\end{shrinkeq}
\begin{shrinkeq}{-1ex}{-1ex}
	\begin{align}
	F_{j1} = C_{j1}H_{j1}L_{j1}^{'} = 0.42\times 22.8\times 10^{2} \times 0.0823 \A = 78.8 \A
	\end{align}
\end{shrinkeq}
\begin{shrinkeq}{-1ex}{-1ex}
	\begin{align}
	\frac{h_{j2}^{'}}{\uptau} = \frac{0.01517}{0.106} = ,\qquad B_{j2} =1.526 \T\qquad\text{于是}C_{j2} = 0.18
	\end{align}
\end{shrinkeq}
\begin{shrinkeq}{-1ex}{-1ex}
	\begin{align}
	F_{j2} = C_{j2}H_{j1}L_{j2}^{'} = 0.18\times 23.1\times 10^{2}\times 0.0268 \A =11.14 \A
	\end{align}
\end{shrinkeq}
\entry
每极磁势
\begin{shrinkeq}{-1ex}{-1ex}
	\begin{align}
	F_{0} &= F_{\delta}+F_{i1}+F_{i2}+F_{j1}+F_{j2} \\
	&=\left(125.3 + 28.44+ 24.82+ 78.8+ 11.14\right)\A = 268.5\A \notag
	\end{align}
\end{shrinkeq}
\entry
按式(3-57)计算满载磁化电流
\begin{shrinkeq}{-1ex}{-1ex}
	\begin{align}
	I_{m} &= \frac{2pF_0}{0.9m_1N_1K_{dq1}} \\
	&= \frac{2\times 1\times 268.5}{0.9\times 3\times 354\times 0.927}\A = 0.606\A\notag
	\end{align}
\end{shrinkeq}
\entry
磁化电流标幺值
\begin{shrinkeq}{-1ex}{-1ex}
	\begin{align}
	I_{m}^{*} = \frac{I_m}{I_{KW}} = \frac{0.606}{1.136} = 0.533
	\end{align}
\end{shrinkeq}
\entry
励磁电抗按式(4-149)计算
\begin{shrinkeq}{-1ex}{-1ex}
	\begin{align}
	X_{ms} &= 4f\mu_0\frac{m_1}{\pi}\frac{\left(N_1K_{dp1}\right)^2}{K_sp}l_{ef}\frac{\uptau}{\delta_{ef}}\\
	&= \frac{4\times 50\times 4\pi\times 10^{-7}\times 3\times\left(354\times 0.927\right)^2\times 0.0655\times 0.106}{\pi\times 1.42\times 1\times 0.268\times 10^{-3}}\Ohm\notag\\
	&= 488.28\Ohm
	\end{align}
\end{shrinkeq}
\begin{shrinkeq}{-1ex}{-1ex}
	\begin{align}
	X_{ms}^{*} = X_{ms}\frac{I_{KW}}{U_{N\phi}} = \frac{488.28\times 1.136}{220} = 2.521
	\end{align}
\end{shrinkeq}



\section{参数计算}
\entry
线圈平均半匝长度(见图附1-4)
\par
定子线圈节距
\begin{shrinkeq}{-1ex}{-1ex}
	\begin{align}
	\uptau_{y} &= \frac{\pi\left[D_{i1}+2\left(h_{01}+h_{11}\right)+h_{21}+r_{21}\right]}{2p}\beta\\
	&=\frac{\pi\left[ + 2\left( + \right)\times 10^{-3}+ \times 10^{-3}+ \times 10^{-3}\right]}{2\times 2} \times  \notag\\
	&= \m \notag
	\end{align}
\end{shrinkeq}
其中节距比$\beta = \sfrac{\left(  \right)}{\left(  \right)} = $式平均值。
\par
直线部分长度
\begin{shrinkeq}{-1ex}{-1ex}
	\begin{align}
	l_{B} = l_{i}+2d_1 = \left(0.040+2\times \right)\m = \m
	&= \m \notag
	\end{align}
\end{shrinkeq}
其中,$d_1$式线圈直线部分伸出铁心的长度,取$10\sim 30\mm$,机座大,极数少者取较大值。
\par
平均半匝长
\begin{shrinkeq}{-1ex}{-1ex}
	\begin{align}
	l_{o} = l_{B} +K_{o}\uptau_{y} = + \times  = \m
	\end{align}
\end{shrinkeq}
其中,$K_{o}$是经验系数,2极取$1.16$,4、6极取$1.2$,8极取$1.25$。
\entry
端部平均长
\begin{shrinkeq}{-1ex}{-1ex}
	\begin{align}
	l_{E} = 2d_1 +K_{o}\uptau_{y} = \left( \times + \times \right)\m = \m
	\end{align}
\end{shrinkeq}
\entry
由式(4-38)可知感应电机定子绕组的漏抗为
\begin{shrinkeq}{-1ex}{-1ex}
	\begin{align}
	X_{\sigma 1} = 4\pi f\mu_0\frac{N_{1}^2}{pq_1}l_{ef}\ssum\lambda_{1}
	\end{align}
\end{shrinkeq}
除以阻抗基值$Z_{KW}=\sfrac{U_{N\phi}}{I_{KW}} = \sfrac{m_1U_{N\phi}^{2}}{P_{N}}$,便可得定子漏抗标幺值
\begin{shrinkeq}{-1ex}{-1ex}
	\begin{align}
	X_{\sigma 1}^{*} = C_x\left(\frac{2m_1p}{Z_1K_{dp1}^{2}}\ssum\lambda_{1}\right)
	\end{align}
\end{shrinkeq}
式中,$\ssum\lambda_{1} = \lambda_{s1}+\lambda_{\delta 1}+\lambda_{E1}$,$C_{x}$为漏抗系数,等于
\begin{shrinkeq}{-1ex}{-1ex}
	\begin{align}
	C_{x} &= \frac{4\pi f\mu_0\left(N_1K_{dp1}\right)^2 l_{ef}P_{N}}{m_1pU_{N\phi}^2} \\
	&= \frac{4\pi\times 50\times 4\pi\times 10^{-7}\times\left(432\times 0.9456\right)^2\times 0.0436\times 750}{3\times 2\times 220^2}\notag\\
	& = 
	\end{align}
\end{shrinkeq}
\entry
按照附录四计算定子槽比漏磁导。因为是双层绕组,整距?,节距漏抗系数$K_{v1} = K_{L1} = $。
\begin{shrinkeq}{-1ex}{-1ex}
	\begin{align}
	\lambda_{s1} = K_{v1}\lambda_{v1}+K_{L1}\lambda_{L1} = \times + \times = 
	\end{align}
\end{shrinkeq}
其中
\begin{shrinkeq}{-1ex}{-1ex}
	\begin{align}
	\lambda_{v1} = \frac{h_{01}}{b_{01}} +\frac{2h_{11}}{b_{01}+b_{11}} = \frac{ }{ } + \frac{ }{ } =  
	\end{align}
\end{shrinkeq}
\begin{shrinkeq}{-1ex}{-1ex}
	\begin{align}
	\lambda_{L1} = \text{,因}\frac{h_{21}}{2r_{21}} = \frac{ - }{2\times } = ,\frac{b_{11}}{2r_{21}} = \frac{ }{2\times } = 
	\end{align}
\end{shrinkeq}
\entry
只在铁心部分由槽漏抗,因而计算槽漏抗时要乘上$\sfrac{l_{i}}{l_{ef}}$:
\begin{shrinkeq}{-1ex}{-1ex}
	\begin{align}
	X_{s1}^{*} &= \frac{2m_1p}{Z_1K_{dp1}^{2}}\cdot\frac{l_{i}}{l_{ef}}\lambda_{s1}C_x\\
	&=\frac{2\times 3\times 2}{36\times 0.9456^2}\times\frac{0.040}{0.0436}\times C_x\notag\\
	&= C_x \notag
	\end{align}
\end{shrinkeq}
\entry
考虑到饱和的影响,定子谐波漏抗可按式(4-76)代入式(4-38)计算:
\begin{shrinkeq}{-1ex}{-1ex}
	\begin{align}
	X_{\delta 1}^{*} &= \frac{2m_1p}{Z_1K_{dp1}^{2}}\lambda_{s1}C_x=\frac{2m_1p}{Z_1K_{dp1}^{2}}\cdot\frac{m_1q_1\uptau}{\pi^2\delta_{ef}K_s}C_x = \frac{m_1}{\pi^2}\frac{\uptau}{\delta_{ef}}\frac{\ssum s}{K_{dp1}^{2}K_{s}}\\
	&=\frac{3}{\pi^2}\times\frac{0.0605}{ }\frac{ }{0.9456^2 \times}C_x = C_x \notag
	\end{align}
\end{shrinkeq}
其中,$\ssum s = $由图4-10以$q_1 = 3$,$\beta = $查出。
\entry
双层叠绕组的端部漏抗与 ,

\entry
定子漏抗标幺值
\begin{shrinkeq}{-1ex}{-1ex}
	\begin{align}
	X_{\sigma 1}^{*} &= X_{s1}^{*}+X_{\delta 1}^{*}+X_{E1}^{*} \\
	&=\left( + + \right)C_x = C_x = 
	\end{align}
\end{shrinkeq}
\entry
转子漏抗标幺值的计算与丁子漏抗标幺值的计算相似,但要将转子漏抗折算到定子边。将转子数据
\begin{shrinkeq}{-1ex}{-1ex}
	\begin{align}
	N_2 =  ,\qquad pq_2 = \frac{Z_2}{2m_2} = 
	\end{align}
\end{shrinkeq}
代入式(4-38),乘以阻抗折算系数
\begin{shrinkeq}{-1ex}{-1ex}
	\begin{align}
	K=\frac{4m_1\left(N_1K_{dp1}\right)^2}{Z_2}
	\end{align}
\end{shrinkeq}
和除以阻抗基值,便有
\begin{shrinkeq}{-1ex}{-1ex}
	\begin{align}
	X_{\sigma 2}^{*} = \frac{2m_1p}{Z_2}\ssum\lambda_2C_x
	\end{align}
\end{shrinkeq}
\entry
转子槽比漏磁导的计算见附录四。
\begin{shrinkeq}{-1ex}{-1ex}
	\begin{align}
	\lambda_{s2} = \lambda_{v2}+\lambda_{L2} =  + = 
	\end{align}
\end{shrinkeq}
其中
\begin{shrinkeq}{-1ex}{-1ex}
	\begin{align}
	\lambda_{v2} = \frac{h_{02}}{b_{02}} = \frac{ }{ } = 
	\end{align}
\end{shrinkeq}
\begin{shrinkeq}{-1ex}{-1ex}
	\begin{align}
	\lambda_{L2} = \frac{2h_{12}}{b_{02}+b_{12}}+\lambda_{L} = \frac{2\times }{ + }+ = ,\qquad\lambda_{L} = \text{由}
	\end{align}
\end{shrinkeq}
\begin{shrinkeq}{-1ex}{-1ex}
	\begin{align}
	\frac{h_{22}}{2r_{22}}  = \frac{ }{2\times } = ,\qquad \frac{b_{12}}{2r_{22}} = \frac{ }{2\times } = \text{查曲线得到。}
	\end{align}
\end{shrinkeq}
\entry
转子槽漏抗标幺值
\begin{shrinkeq}{-1ex}{-1ex}
	\begin{align}
	X_{s2}^{*} &= \frac{2m_1p}{Z_2}\frac{l_{i}}{l_{ef}}\lambda_{s2}C_{x} \\
	& = \frac{2\times 3\times 2}{28}\times \frac{0.040}{0.0436}\times C_x = C_x\notag
	\end{align}
\end{shrinkeq}
\entry
考虑饱和影响的谐波比漏磁导可由式(4-79)求出,于是转子谐波漏抗标幺值
\begin{shrinkeq}{-1ex}{-1ex}
	\begin{align}
	X_{\delta 2}^{*} &= \frac{Z_2}{2p\pi^2}\cdot\frac{\uptau}{\delta_{ef}}\cdot\frac{\ssum R}{K_s}\cdot\frac{2m_1p}{Z_2}C_x=\frac{m_1\uptau\ssum R}{\pi^2\delta_{ef}K_{s}}C_x\\
	&= \frac{3\times 0.0605\times }{\pi\times \times }C_x = C_x\notag
	\end{align}
\end{shrinkeq}
其中,$\ssum R = $由图4-11以$\sfrac{Z_2}{2p}=\frac{28}{4} = 7$查出。
\entry
转子绕组端部比漏磁导按式(4-95)计算,于是转子绕组端部漏抗标幺值
\begin{shrinkeq}{-1ex}{-1ex}
	\begin{align}
	X_{E2}^{*} &= \frac{0.2523Z_2D_R}{2pl_{ef}\times 2p}\frac{2m_1p}{Z_2}C_x = \frac{0.757}{l_{ef}}\cdot\frac{D_R}{2p}C_x \\
	&= \frac{0.757}{0.0436}\times\frac{ }{2\times 2}C_x = C_x\notag
	\end{align}
\end{shrinkeq}
\entry
转子斜槽漏抗按式(4-162)计算
\begin{shrinkeq}{-1ex}{-1ex}
	\begin{align}
	X_{sk}^{*} &= 0.5\left(\frac{b_{sk}}{t_2}\right)^2X_{\delta 2}^2 \\
	&= 0.5\times\left(\frac{ }{0.00858}\right)^2\times  C_x = C_x\notag
	\end{align}
\end{shrinkeq}
\entry
转子漏抗标幺值
\begin{shrinkeq}{-1ex}{-1ex}
	\begin{align}
	X_{\sigma 2}^{*} &= X_{s2}^{*}+X_{\delta 2}^{*}+X_{E2}^{*} +X_{sk}^{*} = \left( + + + \right)C_x \\
	&= C_x = \notag
	\end{align}
\end{shrinkeq}
\entry
定转子漏抗标幺值之和
\begin{shrinkeq}{-1ex}{-1ex}
	\begin{align}
	X_{\sigma}^{*} = X_{\sigma 1}^{*}+X_{\sigma 2}^{*} = + = 
	\end{align}
\end{shrinkeq}
\entry
定子绕组直流电阻按式(4-5)计算
\begin{shrinkeq}{-1ex}{-1ex}
	\begin{align}
	R_1 = \rho_{\omega}\frac{2N_1l_o}{N_{i1}A_{c1}^{'}a_1}=\frac{0.0217\times 10^{-6}\times 2\times 432\times }{1\times 0.328\times 10^{-6}\times 1}\Ohm = \Ohm
	\end{align}
\end{shrinkeq}
其中,$\rho_{\omega}=0.0217\times 10^{-6}\Ohm\cdot m$为B级绝缘平均工作温度$75\celsius$时铜的电阻率。
\entry
定子绕组相电阻标幺值
\begin{shrinkeq}{-1ex}{-1ex}
	\begin{align}
	R_1^{*} = R_1\frac{I_{KW}}{U_{N\phi}} = \frac{ \times 1.136}{220} = 
	\end{align}
\end{shrinkeq}
\entry
有效材料的计算
\par
感应电机的有效材料是指定子绕组导电材料和定转子铁心到此材料,电机的成本主要由有效材料的用量决定(见\textsection2-1)。定子铜的重量
\begin{shrinkeq}{-1ex}{-1ex}
	\begin{align}
	G_{\text{Cu}} &= Cl_{o}N_{s1}Z_{1}A_{c1}^{'}N_{i1}\rho_{\text{Cu}}\\
	&= 1.05\times \times 72 \times 36 \times 0.328\times 10^{-6}\times 1 \times 8.9\times 10^{-3}\kg = \kg\notag
	\end{align}
\end{shrinkeq}
其中,$C$是考虑导线绝缘和引线重量的系数,漆包圆铜线$C=1.05$;$\rho_{\text{Cu}} = 8.9\times 10^{3}\sfrac{\kg}{\m^3}$是铜的密度。
\par
硅钢片重量
\begin{shrinkeq}{-1ex}{-1ex}
	\begin{align}
	G_{\text{Fe}} &= K_{Fe}l_{i}\left(D_1+\delta\right)^2\rho_{\text{F}}\\
	&= 0.95\times 0.040\times \left(0.12+ \right)^2\times 7.8\times 10^{3}\kg =  \kg\notag
	\end{align}
\end{shrinkeq}
其中,$\delta =  \m$是冲剪余量,$\rho_{\text{Fe}} = 7.8\times 10^{3}\sfrac{\kg}{\m^3}$是硅钢片的密度。

\entry
按式(4-12)并这算至定子边便可计算转子电阻的折算值
\begin{shrinkeq}{-1ex}{-1ex}
	\begin{align}
	R_{2}^{'}\approx\rho_{\omega}\left(\frac{K_Bl_B}{A_B}+\frac{Z_2D_R}{2\pi p^2A_{R}}\right)\frac{4m_1\left(N_1K_{dp1}\right)^2}{Z_2} = R_{B}^{'}+R_{R}^{'}
	\end{align}
\end{shrinkeq}
其中,$K_{B}$是考虑铸铝转子因叠片不整齐,造成槽面积减小,导条电阻增加,通常取$K_B = 1.04$。
\begin{shrinkeq}{-1ex}{-1ex}
	\begin{align}
	R_{B}^{'} = \frac{ 0.0434\times 10^{-6}\times 1.04\times }{ }\times\frac{4\times 3\times \left(432\times 0.9456\right)^2}{16}\Ohm = \Ohm
	\end{align}
\end{shrinkeq}
\begin{shrinkeq}{-1ex}{-1ex}
	\begin{align}
	R_{B}^{*} = R_{B}^{'}\frac{I_{KW}}{U_{N\phi}} =\frac{ \times 1.136}{220} = 
	\end{align}
\end{shrinkeq}
\begin{shrinkeq}{-1ex}{-1ex}
	\begin{align}
	R_{R}^{'} = \frac{ 0.0434\times 10^{-6}\times  \times 4\times 3\times\left(432\times 0.9456\right)^2}{2\pi\times 1^2\times }\Ohm = \Ohm
	\end{align}
\end{shrinkeq}
\begin{shrinkeq}{-1ex}{-1ex}
	\begin{align}
	R_{R}^{*} = R_{R}^{'}\frac{I_{KW}}{U_{N\phi}} =\frac{ \times 1.136}{220} = 
	\end{align}
\end{shrinkeq}
其中,$\rho_{\omega} = 0.0434\times 10^{-6}\Omega\cdot\m$是B级绝缘平均工作温度$75\celsius$时铝的电阻率。
\begin{shrinkeq}{-1ex}{-1ex}
	\begin{align}
	R_2^{*} = R_{B}^{*}+R_{R}^{*} = + =
	\end{align}
\end{shrinkeq}
\entry
定子电流有功分量标幺值按式(10-48)计算
\begin{shrinkeq}{-1ex}{-1ex}
	\begin{align}
	I_{1P}^{*} = \frac{1}{\eta^{'}} = 
	\end{align}
\end{shrinkeq}
\entry
转子电流无功分量标幺值按式(10-49)计算
\begin{shrinkeq}{-1ex}{-1ex}
	\begin{align}
	I_{x}^{*} &= \sigma_1 X_{\sigma}^{*}\left(I_{1P}^{*}\right)^2\left[1+\left(\sigma_1X_{\sigma}^{*}I_{1P}^{*}\right)\right]\\
	&= \times \times ^2\times\left[1+\left(\times \times \right)^2\right] \notag\\
	&= \notag
	\end{align}
\end{shrinkeq}
其中系数$\sigma_1$
\begin{shrinkeq}{-1ex}{-1ex}
	\begin{align}
	\sigma_1 = 1+\frac{X_{\sigma_1}^{*}}{X_{ms}^{*}} = 1+\frac{ }{ } = 
	\end{align}
\end{shrinkeq}
\entry
定子电流无功分量标幺值按式(10-47)计算
\begin{shrinkeq}{-1ex}{-1ex}
	\begin{align}
	I_{1Q}^{*} = I_{m}^{*} +I_{X}^{*} = + =
	\end{align}
\end{shrinkeq}
\entry
满载电势标幺值按式(10-6)计算
\begin{shrinkeq}{-1ex}{-1ex}
	\begin{align}
	K_{E} &= 1-\epsilon_{L} = 1-\left(I_{1P}^{*}R_{1}^{*}+I_{1Q}^{*}X_{\sigma_1}^{*}\right)\\
	&= 1-\left( \times + \times \right) = \notag
	\end{align}
\end{shrinkeq}
与第22项的初设值$K_{E}^{'}$相符。
\entry
由式(3-52)计算空载电势标幺值
\begin{shrinkeq}{-1ex}{-1ex}
	\begin{align}
	1-\epsilon_{0} = 1-I_{m}^{*}X_{\sigma_1}^{*} = 1- \times = 
	\end{align}
\end{shrinkeq}
\entry
假设饱和系数$K_{s}$不变,波幅系数$F_{s}$不变,于是空载时定子齿部磁密及磁场强度
\begin{shrinkeq}{-1ex}{-1ex}
	\begin{align}
	B_{i1_0} = \frac{1-\epsilon_{0}}{1-\epsilon_{L}}B_{i1} = \frac{ }{ }\times \T = \T;\qquad H_{i1_0} = \sfrac{\A}{\cm}
	\end{align}
\end{shrinkeq}
\entry
空载时转子齿部磁密及磁场强度
\begin{shrinkeq}{-1ex}{-1ex}
	\begin{align}
	B_{i2_0} = \frac{1-\epsilon_{0}}{1-\epsilon_{L}}B_{i2} = \frac{ }{ }\times \T = \T;\qquad H_{i2_0} = \sfrac{\A}{\cm}
	\end{align}
\end{shrinkeq}
\entry
空载时定子轭部磁密及磁场强度
\begin{shrinkeq}{-1ex}{-1ex}
	\begin{align}
	B_{j1_0} = \frac{1-\epsilon_{0}}{1-\epsilon_{L}}B_{j1} = \frac{ }{ }\times \T = \T;\qquad H_{j1_0} = \sfrac{\A}{\cm}
	\end{align}
\end{shrinkeq}
\entry
空载时转子轭部磁密及磁场强度
\begin{shrinkeq}{-1ex}{-1ex}
	\begin{align}
	B_{j2_0} = \frac{1-\epsilon_{0}}{1-\epsilon_{L}}B_{j2} = \frac{ }{ }\times \T = \T;\qquad H_{j2_0} = \sfrac{\A}{\cm}
	\end{align}
\end{shrinkeq}
\entry
空载气隙磁密
\begin{shrinkeq}{-1ex}{-1ex}
	\begin{align}
	B_{\delta_0} = \frac{1-\epsilon_{0}}{1-\epsilon_{L}}B_{\delta} = \frac{ }{ }\times \T = \T
	\end{align}
\end{shrinkeq}
\entry
空载时定子齿部磁压降
\begin{shrinkeq}{-1ex}{-1ex}
	\begin{align}
	F_{i1_0} = H_{i1_0}L_{i1} = \times \A = \A
	\end{align}
\end{shrinkeq}
\entry
空载时转子齿部磁压降
\begin{shrinkeq}{-1ex}{-1ex}
	\begin{align}
	F_{i2_0} = H_{i2_0}L_{i2} = \times \A = \A
	\end{align}
\end{shrinkeq}
\entry
空载时定子轭部磁压降,此时$C_{j1} = $
\begin{shrinkeq}{-1ex}{-1ex}
	\begin{align}
	F_{j1_0} = C_{j1}H_{i1_0}L_{j1}^{'} = \times \times \A = \A
	\end{align}
\end{shrinkeq}
\entry
空载时转子轭部磁压降,此时$C_{j2} = $
\begin{shrinkeq}{-1ex}{-1ex}
	\begin{align}
	F_{j2_0} = C_{j2}H_{i2_0}L_{j2}^{'} = \times \times \A = \A
	\end{align}
\end{shrinkeq}
\entry
空载时气隙磁压降
\entry
空载时定子轭部磁压降,此时$C_{j1} = $
\begin{shrinkeq}{-1ex}{-1ex}
	\begin{align}
	F_{\delta_0} = \frac{K_{\delta}\delta B_{\delta_0}}{\mu_0} = \frac{ \times \times }{4\pi\times 10^{-7}} \A = \A
	\end{align}
\end{shrinkeq}
\entry
空载时每极磁势
\entry
空载时定子轭部磁压降,此时$C_{j1} = $
\begin{shrinkeq}{-1ex}{-1ex}
	\begin{align}
	F_{0_0} &= F_{\delta_0}+F_{i1_0}+F_{i2_0}+F_{j1_0}+F_{j2_0}\\
	&=\left( + + + + \right)\A = \A
	\end{align}
\end{shrinkeq}
\entry
空载磁化电流
\entry
空载时定子轭部磁压降,此时$C_{j1} = $
\begin{shrinkeq}{-1ex}{-1ex}
	\begin{align}
	I_{m_0} = \frac{2pF_{0_0}}{0.9m_1N_1K_{dp1}} = \frac{2\times }{0.9\times 3\times \times }\A = \A
	\end{align}
\end{shrinkeq}
感应电机的空载电流$I_0$可认为近似等于空载磁化电流。
\par
感应电动机从空载导额定负载,感应电势变化不大,不必计算整条空载特性曲线,只要计算额定负载和空载两种状态下的磁化电流就可以了。



\section{工作性能计算}
\entry
由式(10-46)计算定子电流标幺值
\begin{shrinkeq}{-1ex}{-1ex}
	\begin{align}
	I_{1}^{*} &= \sqrt{I_{1P}^{*2}+I_{1Q}^{*2}} = \sqrt{ ^2+ ^2} = \\
	I_{1} &= I_{1}^{*}I_{KW} = \times \A = \A 
	\end{align}
\end{shrinkeq}
\entry
定子电流密度
\begin{shrinkeq}{-1ex}{-1ex}
	\begin{align}
	J_1 = \frac{I_{1}}{aN_{1}A_{c1}^{'}} = \frac{  }{ \times \times }\sfrac{\A}{\mm^2} = \sfrac{\A}{\mm^2}
	\end{align}
\end{shrinkeq}
\entry
定子线负荷
\begin{shrinkeq}{-1ex}{-1ex}
	\begin{align}
	A_{1} = \frac{m_1N_{\phi 1}I_1}{\pi D_{i1}} = \frac{3\times \times }{\pi\times }\sfrac{\A}{\m} = \sfrac{\A}{\m}
	\end{align}
\end{shrinkeq}
\entry
转子电流标幺值
\begin{shrinkeq}{-1ex}{-1ex}
	\begin{align}
	I_{2}^{*} = \sqrt{I_{1P}^{*2}+I_{X}^{*2}} = \sqrt{ ^2+ ^2} = 
	\end{align}
\end{shrinkeq}
\par
导条电流实际值
\begin{shrinkeq}{-1ex}{-1ex}
	\begin{align}
	I_{2} = I_{2}^{*}I_{KW}\frac{m_1N_{\phi 1}K_{dp1}}{Z_2} = \times 1.136\times\frac{3\times \times }{16}\A = \A
	\end{align}
\end{shrinkeq}
\par
端环电流实际值
\begin{shrinkeq}{-1ex}{-1ex}
	\begin{align}
	I_{R} = I_{2}\frac{Z_{2}}{2\pi p} = \times \frac{16}{2\pi\times 1}\A = \A
	\end{align}
\end{shrinkeq}
\entry
转子电流密度
\par
导条电密
\begin{shrinkeq}{-1ex}{-1ex}
	\begin{align}
	J_{B} = \frac{I_2}{A_{B}} = \frac{ }{ }\sfrac{\A}{\mm^2} = \sfrac{\A}{\mm^2}
	\end{align}
\end{shrinkeq}
\par
端环电密
\begin{shrinkeq}{-1ex}{-1ex}
	\begin{align}
	J_{R} = \frac{I_{R}}{A_{R}} = \frac{ }{ }\sfrac{\A}{\mm^2} = \sfrac{\A}{\mm^2}
	\end{align}
\end{shrinkeq}
\entry
定子铜损耗的标幺值
\begin{shrinkeq}{-1ex}{-1ex}
	\begin{align}
	p_{\text{Cu}_1}^{*} &= \frac{p_{\text{Cu}_1}}{P_{N}}=\frac{m_1I_{1}^{2}R_1}{m_1I_{KW}^{2}Z_{KW}} = I_{1}^{*2}R_{1}^{*}\\
	&= ^2\times  = \notag
	p_{\text{Cu}_1} = p_{\text{Cu}_1}^{*}P_{N} = \times 750\W = \W
	\end{align}
\end{shrinkeq}
\entry
转子铝损耗的标幺值
\begin{shrinkeq}{-1ex}{-1ex}
	\begin{align}
	p_{\text{Al}_2}^{*} &= I_{2}^{*2}R_{2}^{*} = ^2\times = \\
	p_{\text{Al}_2} &= p_{\text{Al}_2}^{*}P_N = \times 750\W = \W
	\end{align}
\end{shrinkeq}
\entry
负载时的附加损耗按规定2极时取$p_{s}^{*} = $
\begin{shrinkeq}{-1ex}{-1ex}
	\begin{align}
	p_s = p_{s}^{*}P_N = \times 750\W = \W
	\end{align}
\end{shrinkeq}
\entry
机械损耗可参考同类型相近对各电机由空载实验分析计算得出的出局,若无数据可参考时,则按式(5-82)计算:
\begin{shrinkeq}{-1ex}{-1ex}
	\begin{align}
	p_{fw}&=\left(\frac{3}{p}\right)^2D_{1}^{4}\times 10^{ } \\
	&=\left(\frac{3}{1}\right)^2\times ^4\times 10^{ }\W = \W\notag\\
	p_{fw}^{*}&=\frac{p_{fw}}{P_N} = \frac{ }{750} = 
	\end{align}
\end{shrinkeq}
\entry
铁损耗
\par
先计算基本铁耗,再乘以经验系数就得到全部铁损耗。
\par
定子轭部铁耗由式(5-15)计算
\begin{shrinkeq}{-1ex}{-1ex}
	\begin{align}
	p_{\text{Fe}j} = k_2p_{hej}G_j = 2\times \times \W = \W
	\end{align}
\end{shrinkeq}
式中,$k_2$按经验取为2;
\par
轭部重量
\begin{shrinkeq}{-1ex}{-1ex}
	\begin{align}
	G_{j} = 4\pi A_{j1}L_{j1}^{'}\rho_{\text{Fe}} = 4\times 1\times \times 7.8\times 10^{3}\kg = \kg
	\end{align}
\end{shrinkeq}
\par
$p_{hej}$是轭部铁损耗系数,根据$B_{j1_0} = \T$从附录六查得$p_{hej} = \sfrac{\W}{\kg}$
\par
定子齿部铁耗由式(5-17)计算
\begin{shrinkeq}{-1ex}{-1ex}
	\begin{align}
	p_{\text{Fe}i} = k_1p_{hei}G_{i} = 2.5\times \times \W = \W
	\end{align}
\end{shrinkeq}
式中,$k_1$对于闭口槽按经验取为$2.5$;
\par
齿部重量
\begin{shrinkeq}{-1ex}{-1ex}
	\begin{align}
	G_{i} = 2p\times A_{i1}L_{i1}\rho_{\text{Fe}} = 2\times 1\times \times \times 7.8\times 10^{3}\kg = \kg
	\end{align}
\end{shrinkeq}
\par
$p_{hei}$是齿部铁损耗系数,根据$B_{i1_0} = \T$从附录六查得$p_{hei} = \sfrac{\W}{\kg}$
\par
于是全部铁损耗
\begin{shrinkeq}{-1ex}{-1ex}
	\begin{align}
	p_{\text{Fe}} &= p_{\text{Fe}j}+p_{\text{Fe}i} = \left( + \right)\W = \W\\
	p_{\text{Fe}}^{*}&=\frac{p_{\text{Fe}} }{P_{N}} = \frac{ }{750} =   
	\end{align}
\end{shrinkeq}
\entry
由式(10-53)可计算总损耗标幺值
\begin{shrinkeq}{-1ex}{-1ex}
	\begin{align}
	\ssum p^{*} &= p_{\text{Cu}_1}^{*}+p_{\text{Al}_2}^{*}+p_{s}^{*}+p_{fw}^{*}+p_{\text{Fe}}^{*}\\
	&= + + + + = \notag  
	\end{align}
\end{shrinkeq}
\entry
输入功率标幺值
\begin{shrinkeq}{-1ex}{-1ex}
	\begin{align}
	P_{N1}^{*} = 1+\ssum P^{*} = 1+ = 
	\end{align}
\end{shrinkeq}
\entry
由式(10-52)可得
\begin{shrinkeq}{-1ex}{-1ex}
	\begin{align}
	\eta = 1-\frac{\ssum p^{*}}{P_{N1}^{*}} = 1-\frac{ }{ } = \%   
	\end{align}
\end{shrinkeq}
\par
$\text{误差}=\sfrac{\left( - \right)}{ } = \%> \%$
\par
效率的假设值太低,由\textsection10-5中分析可知,再次假设$\eta^{'\!'}>\eta$,取$\eta^{'\!'} = \%$,重新计算第66$\sim$95项中的有关各项:
\setcounter{designitem}{65}
\entry
$\eta^{'\!'} = \%$,$I_{1P}^{*} = $
\entry
$I_{X}^{*} = $
\entry
$I_{1Q}^{*} = $
\entry
$1-\epsilon_{L} = $,误差$ = \%$
\setcounter{designitem}{82}
\entry
$I_{1}^{*} = $,$I_1 = \A$
\entry
$J_1 = \sfrac{\A}{\mm^2}$
\entry
$A_1 = \sfrac{\A}{\cm}$
\entry
$I_{2}^{*} = $,$I_{2} = \A$,$I_{R} = \A$
\entry
$J_{B} = \sfrac{\A}{\mm^2}$,$J_{R} = \sfrac{A}{\mm^2}$
\entry
$p_{\text{Cu}_1}^{*} = $,$p_{\text{Cu}_1} = \W$
\entry
$p_{\text{Al}_2}^{*} = $,$p_{\text{Al}_2} = \W$
\setcounter{designitem}{92}
\entry
$\ssum p^{*} = $
\entry
$p_{N1}^{*} = $
\entry
$\eta = \%$,误差$=\frac{\left( - \right)}{ } = \%< \%$
\entry
功率因数
\begin{shrinkeq}{-1ex}{-1ex}
	\begin{align}
	\cos\varphi = \frac{I_{1P}^{*}}{I_{1}^{*}} = \frac{ }{ } =    
	\end{align}
\end{shrinkeq}
\entry
额定转差率由式(10-54)计算
\begin{shrinkeq}{-1ex}{-1ex}
	\begin{align}
	s_{N} &= \frac{p_{\text{Al}_2}^{*}}{1+p_{\text{Al}_2}^{*}+p_{\text{Fe}_r}^{*}+p_{s}^{*}+p_{fw}^{*}} \\
	&=\frac{ }{1+ + + + } = \%\notag 
	\end{align}
\end{shrinkeq}
式中
\begin{shrinkeq}{-1ex}{-1ex}
	\begin{align}
	p_{\text{Fe}_r}^{*} &= \frac{p_{\text{Fe}j_{r}}+p_{\text{Fe}i_r}}{P_N}\\
	&=\dfrac{\left(1-\dfrac{1}{2}\right)\times +\left(1-\dfrac{1}{2.5}\times \right)}{750} =   
	\end{align}
\end{shrinkeq}
\entry
额定转速
\begin{shrinkeq}{-1ex}{-1ex}
	\begin{align}
	n_{N} &= n_1\left(1-s_N\right) = \frac{60f}{p}\left(1-s_N\right)\\
	&=\frac{60\times 2}{1}\times \left(1-  \right)\rpm = \rpm
	\end{align}
\end{shrinkeq}
\entry
最大转矩倍数按式(11-55)计算
\begin{shrinkeq}{-1ex}{-1ex}
	\begin{align}
	T_{m}^{*} &= \frac{1-s_N}{2\left(R_{1}^{*}+\sqrt{R_{1}^{*2}+X_{\sigma}^{*2}}\right)}\\
	&=\frac{1- }{2\sqrt{ ^2 + ^2}} =  \notag
	\end{align}
\end{shrinkeq}

\section{起动性能计算}
\entry
按照式(10-56)假设起动电流
\begin{shrinkeq}{-1ex}{-1ex}
	\begin{align}
	I_{st}^{'} = \left(2.5\sim 3.5\right)T_{m}^{*}I_{KW} = 3\times \times 1.136\A = \A
	\end{align}
\end{shrinkeq}
\entry
起动时产生漏磁的定转子槽磁势平均值由式(10-59)计算
\begin{shrinkeq}{-1ex}{-1ex}
	\begin{align}
	F_{st} &= I_{st}^{'}\frac{N_{s1}}{\sqrt{2}a_1}\left(K_{v1}-K_{d1}^2K_{p1}\frac{Z_1}{Z_2}\right)\sqrt{1-\epsilon_{0}}\\
	&= \times \frac{ }{\sqrt{2}\times }\times\left( + ^2\times \times\frac{18}{16}\right)\times\sqrt{ }\A = \A\notag
	\end{align}
\end{shrinkeq}
由此磁势产生的虚拟磁密按式(10-60)计算
\begin{shrinkeq}{-1ex}{-1ex}
	\begin{align}
	s_{N} &= \frac{p_{\text{Al}_2}^{*}}{1+p_{\text{Al}_2}^{*}+p_{\text{Fe}_r}^{*}+p_{s}^{*}+p_{fw}^{*}} \\
	&=\frac{ }{1+ + + + } = \%\notag 
	\end{align}
\end{shrinkeq}
	
	
	
\end{spacing}

\end{document}
